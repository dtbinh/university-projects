\documentclass[a4paper,11pt,titlepage]{article}
\author{Abrie Greeff\\B.Sc Hons (Rekenaarwetenskap)\\Departement Rekenaarwetenskap\\Universiteit van Stellenbosch}
\date{24 April 2006}
\title{Generiese Vaardighede: gawk oefening}
\begin{document}
\maketitle
\part{Genomedata}
\paragraph{Vraag 1}
\begin{verbatim}
Soek vir die patroon GGTTAA en vertoon die DE lyn van die 
groepe waar die patroon gevind is.
\end{verbatim}
Om dit te doen moet die volgende uitgevoer word.
\begin{verbatim}
gawk -f ../q0.awk genomedata.txt
\end{verbatim}
Die awk script lyk soos volg:
\begin{verbatim}
BEGIN {n = 0; id = ""; de = "";}
{ 
SY = $9
id = $1
if (id ~ /DE/) {
	de = $0
}
if (SY == "") {
	n = 0;
}
if (SY ~ /'G';/) {


	if ((n != 0) && (n != 1)) 
		n = 0
	else
		if (n == 1) 
			n = 2
		else
			if (n == 0) 
				n = 1
	
}
if (SY ~ /'T';/) {
	if ((n != 2) && (n != 3)) 
		n = 0
	else
		if (n == 3) 
			n = 4
		else
			if (n == 2) 
				n = 3
	
}
if (SY ~ /'A';/) {
	if ((n != 4) && (n != 5)) 
		n = 0
	else
		if (n == 5) 
			n = 0
			print de 
		else
			if (n == 4) 
				n = 5
	
}
if (SY ~ /'C';/) {
	n = 0
}
}

\end{verbatim}

\part{Test data}

\paragraph{Vraag 2}
\begin{verbatim}
Orals waar "start" gevind word vervang dit met "START".
\end{verbatim}
Om dit te doen moet die volgende uitgevoer word.
\begin{verbatim}
ls | gawk -f ../q1.awk | csh
\end{verbatim}
Die \emph{directory} waar dit uitgevoer word is een vlak laer as waar die awk script l\^e.
Die awk script lyk soos volg:
\begin{verbatim}
{for (i = 0;i < NF;i++) {
	print "sed -e 's/start/START/g' -i " $i
	}}
\end{verbatim}

\paragraph{Vraag 3}
\begin{verbatim}
Tel hoeveel kere die woord "FINAL" voorkom.
\end{verbatim}
Om dit te doen moet die volgende uitgevoer word.
\begin{verbatim}
cat * | gawk -f ../q2.awk
\end{verbatim}

Die awk script lyk soos volg:
\begin{verbatim}
BEGIN {count = 0}
/FINAL/ {count = count + 1}
END {print count}
\end{verbatim}

\paragraph{Vraag 4}
\begin{verbatim}
Tel hoeveel files bevat minder as vyf lyne teks.
\end{verbatim}
Om dit te doen moet die volgende uitgevoer word.
\begin{verbatim}
ls | gawk -f ../q3.awk | csh | gawk -f q2.awk
\end{verbatim}

Die awk scripts lyk soos volg:
\begin{verbatim}
# q3.awk
{for (i = 0;i < NF;i++) {
	print "gawk -f q3_2.awk " $i
	}}
\end{verbatim}
en
\begin{verbatim}
# q3_2.awk
END { if (NR < 5) print NR "FINAL" }
\end{verbatim}

\paragraph{Vraag 5}
\begin{verbatim}
Verander al die toets files sodat die lyne wat die
woord "FINAL" bevat voorkom net na die lyn wat "START" bevat.
\end{verbatim}
Om dit te doen moet die volgende uitgevoer word.
\begin{verbatim}
ls | gawk -f ../q4.awk | csh
\end{verbatim}

Die awk scripts lyk soos volg:
\begin{verbatim}
# q4.awk
{for (i = 0;i < NF;i++) {
	print "echo Changing " $i
	print "gawk -f ../q4_2.awk " $i " > temp"
	print "cat temp > " $i
	}
	print "rm -f temp"}

\end{verbatim}
en
\begin{verbatim}
# q4_2.awk
{n=0}
/START/ {teks = $0; n = 1}
/FINAL/ {teks = teks "\n" $0; n = 1}
n==0 {teks2 = teks2 "\n" $0;n=0}
END {print teks,teks2}
\end{verbatim}
\end{document}